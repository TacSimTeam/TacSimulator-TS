\chapter{\tacsim の概要}

\tacsim は, 徳山高専で開発されている教育用16bitコンピュータ\tac をパソコン上で再現するソフトウェアである. 本来であれば, \tac を使うためには実機が必要だが, シミュレータを使うことで実機が無くても\tac の演習を行うことができる. また, \ts 言語で記述されており, \vhdl で記述された実機よりも機器や機構の追加が容易である.

\section{目標}

\begin{description}
    \item[ \tac の動作再現 ] \tacsim は, 最終的に授業の演習や\tacos の開発などを手助けするツールとして利用できるようにしたい. そのため, 実機との動作の違いなどを無くすことを目標とする.
    \item[ 拡張容易性 ] \tac 実機は\vhdl で記述されており, ハードウェア化するためにタイミングや回路規模等の制約が多く, 新しい機能を追加するためのコストが大きい. 一方で, ソフトウェアで実現される\tacsim は改造が容易なので新しい機能の検証を手軽に行うことができる.
    \item[ デバッグ用機能 ] \tac の検証のために, デバッグ用の機能を追加することもできる. 例えば, 特定のプロセスを実行している間に何回命令が実行されたかを計測する機能など, 実機に搭載するのが難しい機能についても, シミュレータならば実現可能である. 
    \item[ ブラウザ上での動作 ] \tacsim は \ts 言語で記述されているが, これは将来的にブラウザ上で\tac を利用できるようにするためである. これにより, \tac を所有していない人が\tac がどのように動作するのかを気軽に知ることができる. また, 実機が故障した場合の代替手段としても利用できる.
\end{description}

\section{開発環境}

2023年1月現在の\tacsim は, \electron というソフトウェアフレームワークを使用して開発されている. これにより, Webアプリケーションの開発と同じように\html , \css , \js を使用してスタンドアロンアプリを開発することができる. また, 使用言語には\js の静的な型付けスーパーセットである\ts を使用している.

\section{開発記録}

\begin{description}
    \item[令和2年度卒業研究] \js を用いて\tac のシミュレータの開発を開始. CPUとメモリ部分の実装が終了
    \item[令和3年度卒業研究] \electron を導入し, スタンドアロンアプリとして\tac のシミュレータの開発を行う. 割込みコントローラやI/O機器などの実装が完了し, IPLプログラムが動作するようになった.
    \item[令和4年度卒業研究] 使用言語を\ts に変更. MMUの実装やその他バグの修正, 設計の見直しを行い, 仮想記憶を使用して動作する\tacos が起動し動作するようになった.
\end{description}

\section{実機との違い}

\tacsim には\tac 実機といくつか違う仕様が存在する

\begin{itemize}
    \item \tac 実機とブレークポイントの挙動が異なる. 実機ではCPUが指定したアドレスにアクセスしたときに停止するが, シミュレータではPCが指定したアドレスになったときに停止する. また, ブレークする番地はシミュレータの左下にあるテキストボックスから入力するため, 実機のようにMAレジスタの値をブレークポイントとして使用することはできない.
    \item 実機はRESETボタンを押すと自動的に実行されるが, シミュレータではRUNボタンを押す必要がある.
    \item 実機では, リセット時にSETAボタンが押されていた場合に"\bs kernel.bin"ファイルの代わりに"\bs kernel0.bin"ファイルを読み込む機能があるが, シミュレータには無い.
    \item \tac のタイマーは1ms毎にカウントを進めるが, シミュレータでは動作環境の都合上遅延が発生するため, 正確に1ms毎にカウントが進むわけではない.
    \item 実機とI/Oマップが異なる. 詳しくは付録\ref{appB}を参考のこと.
\end{itemize}