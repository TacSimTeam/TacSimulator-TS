\chapter{\tacsim の概要}

\tacsim は, 徳山高専で開発されている教育用16bitコンピュータ\tac をパソコン上で再現しているソフトウェアである. 本来であれば, \tac を使うためには実機が必要だが, シミュレータを使うことで実機が無くても\tac の演習を行うことができる. また, \ts 言語で記述されており, VHDLで記述された実機よりも機器や機構の追加が容易である.

\section{目標}

\tacsim の開発は以下の目標のもと行われている.

\begin{description}
    \item[ \tac の動作再現 ] \tacsim は最終的に, 授業の演習や\tacos の開発などを手助けするツールとして利用できるようにしたい. そのため, 実機との動作の違いなどを無くすことを目標とする.
    \item[ 拡張容易性 ] \tac 実機はVHDLで記述されており, 新しい機能を追加することためのコストが大きい. そこで, \tacsim を使えばハードウェアを触らなくても新しい機能の検証を行うことができる. \tacsim はオープンソースのプロジェクトなので, 各自が機能を追加することは\tac 実機と比べて難しくない.
    \item[ デバッグ用機能 ] \tac の検証のために, デバッグ用の機能を追加することもできる. 例えば, 特定のプロセスを実行している間に何回命令が実行されたかを計測する機能など, 実機に搭載するのが難しい機能についても, シミュレータならば実現可能である. 
    \item[ ブラウザ上での動作 ] \tacsim は \ts 言語で記述されているが, これは将来的にブラウザ上で\tac を利用できるようにするためである. これにより, 学外の人間が\tac がどのように動作するのかを気軽に知ることができる. また, ハードウェア的な制約を受けなくなるので, 実機が故障した場合の代替手段としても利用できる.
\end{description}

\section{開発環境}

2023年1月現在の\tacsim は, Electronというソフトウェアフレームワークを使用して開発されている. これにより, Webアプリケーションを開発するのと同じようにスタンドアロンアプリを開発することができる. また, 使用言語にはTypeScriptを使用し, 静的型付けによるバグの早期発見などの恩恵を得る.

\section{開発記録}

\begin{description}
    \item[平成20年度卒業研究] JavaScriptを用いて\tec のシミュレータを開発.
    \item[令和2年度卒業研究] JavaScriptを用いて\tac のシミュレータの開発を開始. CPUとメモリ部分の実装が終了
    \item[令和3年度卒業研究] Electronを導入し, スタンドアロンアプリとして\tac のシミュレータの開発を行う. 割込みコントローラやI/O機器などの実装が完了し, IPLプログラムが動作するようになった.
    \item[令和4年度卒業研究] 使用言語をTypeScriptに変更. MMUの実装やその他バグの修正, 設計の見直しを行い, 仮想記憶を使用して動作するTacOSが起動し動作するようになった.
\end{description}